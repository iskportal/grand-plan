%! Author = kid-a
%! Date = 9/30/25

% Preamble
\documentclass[11pt]{book}

% Packages
\usepackage{amsmath}
\usepackage[top=2cm,bottom=2cm,left=2cm, right=2cm]{geometry}
\usepackage{hyperref}
\usepackage{fancyhdr}

\title{Grand Plan}
\author{Independent Society of Knowledge}
\date{last update \today}

% Document
\begin{document}
    \maketitle
    \tableofcontents
    \newpage


    \chapter{About This Document}\label{ch:about-this-document}
    \textbf{Independent Society of Knowledge's Grand Plan} (code name: \textit{Axiom}), represents the most comprehensive attempt to re-imagine and re-construct the entire infrastructure of human knowledge creation, validation, and distribution using modern, decentralized, and open-collaborative technologies.

\section*{About This Repository}

The repository functions on multiple levels: as a systems engineering document detailing the interconnected software platforms and technologies required; as a strategic roadmap outlining the phased implementation of a complete ecosystem; and as a vision statement articulating how decentralized, open-source principles can be applied to solve fundamental problems in academic research and education.

Structured as both a technical deep-dive and accessible overview, the document bridges the gap between abstract ideals of knowledge democratization and concrete implementation details.
It provides stakeholders—from developers and researchers to investors and institutions—with a clear understanding of how each component contributes to the larger mission while maintaining technical rigor in its specifications.

The document is designed to evolve as a living blueprint, incorporating feedback from the academic community, technological developments, and real-world implementation learnings.
It serves not just as a plan, but as a collaborative framework for building consensus around a new paradigm for knowledge work.

\section*{Acknowledgements}

This document is provided with the effort of members of Independent Society of Knowledge and Independent Society of Knowledge.

\section*{Contact}

You are open to discuss topics in this repository in either:
\begin{itemize}
    \item GitHub Issues
    \item Contacting the maintainer: \href{mailto:thisismeamir@outlook.com}{Amir H. Ebrahimnezhad}
\end{itemize}



    \chapter{Introduction}\label{ch:introduction}
    Independent Society of Knowledge, established under the principles of open-collaboration, and free-market for human produced knowledge.
We believe that the current state of academia has problems that potentially, or actively interrupt the flow, and creation of knowledge in the landscape of the community.

In this chapter we would briefly cover our understanding of the situation, problems and the solutions that we provide to this topic.
In the first section you get a general introduction to our vision, and our method of solving the problem while the next section covers a critique to the current state of academia, and reasons we believe in our plan.


\section{Executive Summary}\label{sec:executive-summary}
\subsection{Vision}\label{subsec:vision}
Academia is the pillar of modern civilization, and the future of humankind.
It serves as the source of human produced knowledge, science, technology and is generally considered a vital organ for any society to perform.
Despite this, little to no change has been made in the methodology, implementation, and strategies that govern the academic community for over centuries.

This observation led us to consider an alternative path, that can co-exist along-side the traditional academia, while trying to solve most of the key problems in current state of academia.
These problems vary from research experience, education, review and validation, and finance.
Another important observation was to perceive how other communities had solve similar problems that the scientific community is facing right now.

\textbf{Independent Society of Knowledge} is the collaboration of scientists, researchers, and contributors, with the vision of \textit{Academia as a Decentralized, Open-Collaborative, Free-Market}.
The intention behind ISK is to not only address the problems, but provide solutions, these solutions can have diverse implementations, such as development of platforms, production of licenses and attribution strategies, advocating, and free-open-collaborative goods of economic values.

The approach of Independent Society of Knowledge reflects its vision and belief.
We develop most of the services as free open-source software (FOSS), and advocate for open-collaborative methods, but unlike other movements, we believe that research as a job needs not only attribution, but also money.
This realization would set Independent Society of Knowledge in a sweet-spot, as we're not just advocating for open-collaboration in knowledge (open-education, open-research, open-data, etc.); we also believe in free-market and want to achieve one in the real of knowledge production.
This interchanging vision is because we believe that to achieve a sustainable open-collaboration, the members should be financially safe.

To summarize, Independent Society of Knowledge envisions a model for academia that supports decentralization, open collaboration, and the values of a free market, without losing authenticity and rigorous evaluation of knowledge.
This model seeks to create a more fair, transparent, and accessible ecosystem where contributions are rewarded based on merit, rather than institutional prestige or geographic location.
It fosters an environment where knowledge flows freely across borders, enabling faster scientific progress, broader participation, and equitable access to research and education.
By decentralizing control, empowering individuals, and utilizing modern technologies, ISK aims to reshape the academic landscape to be more inclusive, dynamic, and responsive to the needs of a global, interconnected society.

\subsection{Mission Statement}\label{subsec:mission-statement}
Achieving the vision above requires careful investigation, community advocacy, and development of tools, the mission of Independent Society of Knowledge is therefore, to make this vision real by development of solutions that empower such ideas.

We believe that to solve the problems of academia, one must solve it as a whole and not by little solutions for local problems.
Therefore, our mission is not merely a set of solutions, but a re-imagination of the ecosystem of academia in general.

Thus, our mission is to:
\begin{itemize}
    \item Advocate for open-collaborative, decentralized, and free-market principles in academia
    \item Develop tools and platforms that facilitate open collaboration, peer review, and dissemination of knowledge
    \item Create sustainable financial models that reward contributors based on merit and impact
    \item Foster a global community of researchers, educators, and learners committed to these principles
    \item Promote transparency, inclusivity, and accessibility in all aspects of academic work
\end{itemize}
By pursuing this mission, Independent Society of Knowledge aims to transform the academic landscape into one that is more equitable, dynamic, and responsive to the needs of a rapidly changing world.


\section{Key Objectives}\label{sec:key-objectives}
\subsection{Platforms and Community}\label{subsec:platforms-and-community}
Science and research is a collaborative effort, and the community is the backbone of any scientific endeavor.
One of the key objectives of Independent Society of Knowledge is to provide the means of collaboration, peer-review, and community engagement in decentralized manner for the knowledge community.
This objective would be achieved using a combination of platforms and technologies, such as:
\begin{itemize}
    \item Koncept (Open-Journal): A platform for publishing, reviewing, and discussion of research projects, with versionized, branching model that preserves the history of changes, and allows for collaborative editing and improvement of research works.
    \item Mithra (Social Media): A social-media platform for communication in research, education, and lectures, that supports various modes of communication, such as text, audio, video, and interactive content.
\end{itemize}
Our vision holds that these platforms should be open-source, and decentralized, but interconnected.
This would be achieved by hosting the platforms in a decentralized manner through universities, institutions, and individuals, while also providing a central hub for discovery and interaction.
This objective is crucial for the success of Independent Society of Knowledge, as it would provide the means for the community to thrive, and for the knowledge to flow freely.

\subsection{Library and Computational Ecosystem}\label{subsec:library-and-computational-ecosystem}
Independent Society of Knowledge aims to create an ecosystem of software in real of computational research, data science, and machine learning.
This ecosystem would be different from other existing libraries in three key aspects:
\begin{enumerate}
    \item Libraries are build on top of one another, and are interconnected.
    This would allow for a more modular and flexible approach to software development in scientific realm, where users can easily combine and extend existing libraries to create new functionalities.
    \item Libraries come with set of tool for passive areas of computational science.
    For example, high-performance computing, which makes sure that computational projects have good development experience (DX) and performance.
    \item Our libraries have integration with other projects and source codes in mind, keeping them relative even for people adapting them over time.
\end{enumerate}

Our library ecosystem starts from the Kompute (library for foundational computations) and extends to various domains, such as machine learning, data science, and scientific computing.
This ecosystem would be open-source, and community-driven, allowing for contributions from researchers, developers, and users.
This objective is crucial for the success of Independent Society of Knowledge, as it would provide the means for researchers to perform cutting-edge research, and for the community to build upon existing knowledge.

Another observation in different fields of computational research is that each library is design for the specific job of its developers, therefore, new algorithms, methods, and techniques are often not implemented in these libraries.
Our vision for our library ecosystem is to be able to provide the state-of-the-art algorithms, methods, and techniques in various fields of computational research, as soon as they get published, making our libraries efficient to use and up-to-date, as well as being complete.

\subsection{Education and Learning}\label{subsec:education-and-learning}
Education of the next generation of researchers, scientists, and contributors is crucial for the success of Independent Society of Knowledge.
We make sure to provide the essentials and foundations that would lead the community to open-education.
There are multiple projects in this field such as MIT Open-Courseware, Khan Academy, Coursera, edX, etc.
We seek collaboration and contribute to these ideas as well as promoting a more dynamical approach to education.

\subsection{Licenses and Attribution}\label{subsec:licenses-and-attribution}
One of the key objectives of Independent Society of Knowledge is to provide a set of licenses and attribution strategies that would allow for open-collaborative work, while also providing the means for contributors to be rewarded for their work.
These licenses and attribution strategies would be different from existing ones, such as Creative Commons, GNU General Public License, etc.
Our approach would be to create these licenses and attribution strategies in a way that they are compatible with academic and research endeavours.

Legal aspects of knowledge, maintenance, and attribution is an important mission for Independent Society of Knowledge, and we seek to not overlook this aspect.

\subsection{Finance and Sustainability}\label{subsec:finance-and-sustainability}
\begin{quotation}
    This part needs refinement
\end{quotation}
Financial sustainability is crucial for the success of Independent Society of Knowledge.
While we believe creating Open-Source software for science, research, and education is crucial, we preserve the fact that to make such infrastructure work and maintainable in long-term, we need to provide financial safety for us and our contributors.

Therefore, we look up to a financial model that is based on:
\begin{itemize}
    \item Donations and Grants: We seek donations from individuals, institutions, and organizations that support our vision and mission.
    We also look for grants from governmental and non-governmental organizations that support open-collaborative and decentralized initiatives.
    \item Memberships and Subscriptions: We provide memberships and subscriptions for individuals and institutions that want to support our vision and mission.
    These memberships and subscriptions would provide access to exclusive content, services, and features.
    \item Services and Consulting: We provide services and consulting for individuals, institutions, and organizations that want to implement our vision and mission in their own work.
    These services and consulting would provide expertise, guidance, and support.
\end{itemize}


\section{Impact}\label{sec:impact}
Independent Society of Knowledge is the first of its kind, and has the potential to revolutionize the way we think about academia, research, and education.
We believe our state-of-the-art solutions and the ecosystem of ISK goods can have a significant impact on the academic community, and the world at large.

To give a few examples:
\begin{itemize}
    \item Democratization of Knowledge: By providing open-collaborative platforms, tools, and resources, we can make knowledge more accessible to everyone, regardless of their geographic location, financial status, or institutional affiliation.
    This would lead to a more equitable and inclusive academic community, where everyone has the opportunity to contribute and benefit from knowledge.
    \item Acceleration of Scientific Progress: By fostering a culture of open collaboration and rapid dissemination of research, we can accelerate the pace of scientific discovery and innovation.
    This would lead to faster development of new technologies, treatments, and solutions to global challenges.
    \item Empowerment of Researchers: By providing tools and platforms that facilitate collaboration, peer review, and dissemination of research, we can empower researchers to take control of their own work and careers.
    This would lead to a more dynamic and responsive academic community, where researchers are able to pursue their own interests and goals.
    \item Promotion of Transparency and Reproducibility: By advocating for open data, open methods, and open peer review, we can promote transparency and reproducibility in research.
    This would lead to a more trustworthy and reliable academic community, where research is held to high standards of quality and integrity.
\end{itemize}
In summary, Independent Society of Knowledge has the potential to create a more fair, transparent, and accessible academic ecosystem, where knowledge flows freely, and contributions are rewarded based on merit and impact.



    \chapter{Critique}\label{ch:critique}
    In this chapter we would carefully analyze, and examine the current state of research, academia, and the problems that we believe are present in the current system.
These critiques can lead us to more clearance on the ideas and vision that would be explained in further chapters.


\section{Critique of Current Academia}\label{sec:critique-of-current-academia}

Current state of academia, despite its accomplishments has not changed over the centuries.
Despite the advancement in technologies (provided by academia itself), the system of research and collaboration has not changed in significant ways.

Breaking down the academic and research system to its core components, we can categorize it into the following parts:
\begin{itemize}
    \item Research
    \item Education
    \item Publishing and Dissemination
\end{itemize}

\subsection{Critique of Research}\label{subsec:critique-of-research}

Research is the act of systematic investigation and study in order to discover new knowledge, or to validate existing knowledge.
It is the backbone of modern societies, new technologies, and advancements in our understanding of the world.

This crucial component of society acts in pretty much the same manner that it did centuries ago.
The process of research often requires teams to collaborate, share data, and build upon each other's work.

By the advancement of computers, internet, and digital tools, the process of research has become more efficient, and collaborative.
But at the same time the core functionality of it remains untouched.
We still share papers in unchangeable, un-versionized, documents, that cannot be investigated more than what is written in them.
Although the research is shared it is hard to understand how the community (scientific community) has received it, and what are the critiques, and reviews of it.

Beside these, the process of peer-review remains a black-box, journal-oriented, and slow leading to less transparency, and less efficiency.

There are three main problems to be addressed in research:
\begin{itemize}
    \item Journals and Peer-Reviews
    \item Encapsulated Communication
    \item Lack of Technologies
\end{itemize}

\subsubsection{Journals and Peer-Reviews}\label{subsubsec:journals-and-peer-reviews}
\begin{quotation}
    This section requires extension
\end{quotation}
Journals are black-box of granting prestige and validity of research, and for that they cost money for both the authors, and the readers.
This dysfunctionality arises not because of the journals themselves, but because of the lack of better alternatives.

The traditional process of publishing knowledge has been the cornerstone of science but alternatives can change it for the better.

Two main problems to address here are: (1) Peer-review can be, and is already done via the effort of the respective community, which can in return validate research at a faster pace, with better quality;
(2) Journals gain financially from both the authors, and the readers, and this by removing this intermediate step through community effort, the financial advantage can be granted to the ones who are producing the knowledge themselves.

\subsubsection{Encapsulated Communication}\label{subsubsec:encapsulated-communication}
\begin{quotation}
    This section requires extension
\end{quotation}

Knowledge is currently shared through static, uncollaborative, and unversionized documents.
Tracking advancements, critiques, and reviews of a research is hard, and in some cases impossible.

This encapsulation of communication results in a less active, more localized efforts of producing knowledge, which in return would lower efficiencies of research as a whole.

\subsubsection{Lack of Technologies}\label{subsubsec:lack-of-technologies}
\begin{quotation}
    This section requires extension
\end{quotation}

The problems addressed in previous sections, can be looked more generally.
These problems occurred (to be more specific these aren't problems per se, but inefficiencies that has been noticed because of the modernity we enjoy today) because of the lack of technologies, platforms, and groups that would enhance the processes of research.

One of the key developments that happened over the last centuries was the invention of internet and later on projects like ArXiv.
Although these projects enhanced the way we share and disseminate knowledge, they aren't perfect, and there is still a lot of room for improvement.
The lack of better platforms, and tools that would enhance the process of research, is one of the main problems that we believe is present in the current system.
This lack of better tools, and platforms, is one of the main reasons that we believe the current system is inefficient, and not as effective as it could be.

    \chapter{Decentralization}\label{ch:decentralization}
    This section requires extension.
    \chapter{Open Collaboration}\label{ch:open-collaboration}
    This section requires extension.
    \chapter{Free-Market for Knowledge}\label{ch:free-market-for-knowledge}
    This section requires extension.

    \chapter{Ecosystem}\label{ch:ecosystem}
    Independent Society of Knowledge (ISK) aims to solve the solutions mentioned in \autoref{ch:introduction} and \autoref{ch:critique} by providing a comprehensive ecosystem of goods and services that facilitate open collaboration, decentralization, and a free-market for knowledge.

    The ISK ecosystem, consists of platforms, software tools, initiative, and services that would allow the scientific community to thrive in an open, collaborative, and decentralized environment, while keeping the core values of rigorous scientific methods, and high-quality standards.

    In this chapter, we would briefly cover the main components of the ecosystem, how they connect to one another, and how they contribute to the overall vision of ISK\@.

    \section{Grand View}\label{sec:grand-view}
    To achieve such a vision, multiple services are needed, active and passive initiatives, and tools that would allow the community to gain momentum in the direction of decentralization, open-collaboration, and free-market.
    Our passive initiatives include:
    \begin{itemize}
        \item Advocacy for open-collaboration, decentralization, and free-market for knowledge.
        \item Education and awareness campaigns to inform the community about the benefits of our vision.
        \item Building partnerships with like-minded organizations and institutions to promote our vision.
        \item Licensing and Attribution methods that would allow researchers to retain their rights, while also allowing others to build upon their work.
    \end{itemize}
    Our active initiatives include:
    \begin{itemize}
        \item Koncept: An Integrated Research Environment (IRE) that provides a suite of tools for research at individual level.
        While also providing a new type of document within the academic system that is dynamic, and makes researches more open for collaboration, and branching.
        \item Open-Journal: A decentralized, open-access journal platform that allows researchers to publish their work without the need for traditional publishers.
        This also includes a social platform for researchers that facilitates collaboration, networking, and knowledge sharing.
        The platform would include Koncept integration, Open-Lectures, and other tools to embrace researching experience, as well as communication and collaboration.
        \item Community-Integration: Our projects need to be integrated with other solutions in the community as well.
        Projects like ArXiv, Zenodo, and others that provide open-access to research outputs.
    \end{itemize}

\end{document}
