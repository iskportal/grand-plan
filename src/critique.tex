In this chapter we would carefully analyze, and examine the current state of research, academia, and the problems that we believe are present in the current system.
These critiques can lead us to more clearance on the ideas and vision that would be explained in further chapters.


\section{Critique of Current Academia}\label{sec:critique-of-current-academia}

Current state of academia, despite its accomplishments has not changed over the centuries.
Despite the advancement in technologies (provided by academia itself), the system of research and collaboration has not changed in significant ways.

Breaking down the academic and research system to its core components, we can categorize it into the following parts:
\begin{itemize}
    \item Research
    \item Education
    \item Publishing and Dissemination
\end{itemize}

\subsection{Critique of Research}\label{subsec:critique-of-research}

Research is the act of systematic investigation and study in order to discover new knowledge, or to validate existing knowledge.
It is the backbone of modern societies, new technologies, and advancements in our understanding of the world.

This crucial component of society acts in pretty much the same manner that it did centuries ago.
The process of research often requires teams to collaborate, share data, and build upon each other's work.

By the advancement of computers, internet, and digital tools, the process of research has become more efficient, and collaborative.
But at the same time the core functionality of it remains untouched.
We still share papers in unchangeable, un-versionized, documents, that cannot be investigated more than what is written in them.
Although the research is shared it is hard to understand how the community (scientific community) has received it, and what are the critiques, and reviews of it.

Beside these, the process of peer-review remains a black-box, journal-oriented, and slow leading to less transparency, and less efficiency.

There are three main problems to be addressed in research:
\begin{itemize}
    \item Journals and Peer-Reviews
    \item Encapsulated Communication
    \item Lack of Technologies
\end{itemize}

\subsubsection{Journals and Peer-Reviews}\label{subsubsec:journals-and-peer-reviews}
\begin{quotation}
    This section requires extension
\end{quotation}
Journals are black-box of granting prestige and validity of research, and for that they cost money for both the authors, and the readers.
This dysfunctionality arises not because of the journals themselves, but because of the lack of better alternatives.

The traditional process of publishing knowledge has been the cornerstone of science but alternatives can change it for the better.

Two main problems to address here are: (1) Peer-review can be, and is already done via the effort of the respective community, which can in return validate research at a faster pace, with better quality;
(2) Journals gain financially from both the authors, and the readers, and this by removing this intermediate step through community effort, the financial advantage can be granted to the ones who are producing the knowledge themselves.

\subsubsection{Encapsulated Communication}\label{subsubsec:encapsulated-communication}
\begin{quotation}
    This section requires extension
\end{quotation}

Knowledge is currently shared through static, uncollaborative, and unversionized documents.
Tracking advancements, critiques, and reviews of a research is hard, and in some cases impossible.

This encapsulation of communication results in a less active, more localized efforts of producing knowledge, which in return would lower efficiencies of research as a whole.

\subsubsection{Lack of Technologies}\label{subsubsec:lack-of-technologies}
\begin{quotation}
    This section requires extension
\end{quotation}

The problems addressed in previous sections, can be looked more generally.
These problems occurred (to be more specific these aren't problems per se, but inefficiencies that has been noticed because of the modernity we enjoy today) because of the lack of technologies, platforms, and groups that would enhance the processes of research.

One of the key developments that happened over the last centuries was the invention of internet and later on projects like ArXiv.
Although these projects enhanced the way we share and disseminate knowledge, they aren't perfect, and there is still a lot of room for improvement.
The lack of better platforms, and tools that would enhance the process of research, is one of the main problems that we believe is present in the current system.
This lack of better tools, and platforms, is one of the main reasons that we believe the current system is inefficient, and not as effective as it could be.