\subsection{Vision}\label{subsec:vision}
Academia is the pillar of modern civilization, and the future of humankind.
It serves as the source of human produced knowledge, science, technology and is generally considered a vital organ for any society to perform.
Despite this, little to no change has been made in the methodology, implementation, and strategies that govern the academic community for over centuries.

This observation led us to consider an alternative path, that can co-exist along-side the traditional academia, while trying to solve most of the key problems in current state of academia.
These problems vary from research experience, education, review and validation, and finance.
Another important observation was to perceive how other communities had solve similar problems that the scientific community is facing right now.

\textbf{Independent Society of Knowledge} is the collaboration of scientists, researchers, and contributors, with the vision of \textit{Academia as a Decentralized, Open-Collaborative, Free-Market}.
The intention behind ISK is to not only address the problems, but provide solutions, these solutions can have diverse implementations, such as development of platforms, production of licenses and attribution strategies, advocating, and free-open-collaborative goods of economic values.

The approach of Independent Society of Knowledge reflects its vision and belief.
We develop most of the services as free open-source software (FOSS), and advocate for open-collaborative methods, but unlike other movements, we believe that research as a job needs not only attribution, but also money.
This realization would set Independent Society of Knowledge in a sweet-spot, as we're not just advocating for open-collaboration in knowledge (open-education, open-research, open-data, etc.); we also believe in free-market and want to achieve one in the real of knowledge production.
This interchanging vision is because we believe that to achieve a sustainable open-collaboration, the members should be financially safe.

To summarize, Independent Society of Knowledge envisions a model for academia that supports decentralization, open collaboration, and the values of a free market, without losing authenticity and rigorous evaluation of knowledge.
This model seeks to create a more fair, transparent, and accessible ecosystem where contributions are rewarded based on merit, rather than institutional prestige or geographic location.
It fosters an environment where knowledge flows freely across borders, enabling faster scientific progress, broader participation, and equitable access to research and education.
By decentralizing control, empowering individuals, and utilizing modern technologies, ISK aims to reshape the academic landscape to be more inclusive, dynamic, and responsive to the needs of a global, interconnected society.

\subsection{Mission Statement}\label{subsec:mission-statement}
Achieving the vision above requires careful investigation, community advocacy, and development of tools, the mission of Independent Society of Knowledge is therefore, to make this vision real by development of solutions that empower such ideas.

We believe that to solve the problems of academia, one must solve it as a whole and not by little solutions for local problems.
Therefore, our mission is not merely a set of solutions, but a re-imagination of the ecosystem of academia in general.

Thus, our mission is to:
\begin{itemize}
    \item Advocate for open-collaborative, decentralized, and free-market principles in academia
    \item Develop tools and platforms that facilitate open collaboration, peer review, and dissemination of knowledge
    \item Create sustainable financial models that reward contributors based on merit and impact
    \item Foster a global community of researchers, educators, and learners committed to these principles
    \item Promote transparency, inclusivity, and accessibility in all aspects of academic work
\end{itemize}
By pursuing this mission, Independent Society of Knowledge aims to transform the academic landscape into one that is more equitable, dynamic, and responsive to the needs of a rapidly changing world.