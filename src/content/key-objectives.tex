\subsection{Platforms and Community}\label{subsec:platforms-and-community}
Science and research is a collaborative effort, and the community is the backbone of any scientific endeavor.
One of the key objectives of Independent Society of Knowledge is to provide the means of collaboration, peer-review, and community engagement in decentralized manner for the knowledge community.
This objective would be achieved using a combination of platforms and technologies, such as:
\begin{itemize}
    \item Koncept (Open-Journal): A platform for publishing, reviewing, and discussion of research projects, with versionized, branching model that preserves the history of changes, and allows for collaborative editing and improvement of research works.
    \item Mithra (Social Media): A social-media platform for communication in research, education, and lectures, that supports various modes of communication, such as text, audio, video, and interactive content.
\end{itemize}
Our vision holds that these platforms should be open-source, and decentralized, but interconnected.
This would be achieved by hosting the platforms in a decentralized manner through universities, institutions, and individuals, while also providing a central hub for discovery and interaction.
This objective is crucial for the success of Independent Society of Knowledge, as it would provide the means for the community to thrive, and for the knowledge to flow freely.

\subsection{Library and Computational Ecosystem}\label{subsec:library-and-computational-ecosystem}
Independent Society of Knowledge aims to create an ecosystem of software in real of computational research, data science, and machine learning.
This ecosystem would be different from other existing libraries in three key aspects:
\begin{enumerate}
    \item Libraries are build on top of one another, and are interconnected.
    This would allow for a more modular and flexible approach to software development in scientific realm, where users can easily combine and extend existing libraries to create new functionalities.
    \item Libraries come with set of tool for passive areas of computational science.
    For example, high-performance computing, which makes sure that computational projects have good development experience (DX) and performance.
    \item Our libraries have integration with other projects and source codes in mind, keeping them relative even for people adapting them over time.
\end{enumerate}

Our library ecosystem starts from the Kompute (library for foundational computations) and extends to various domains, such as machine learning, data science, and scientific computing.
This ecosystem would be open-source, and community-driven, allowing for contributions from researchers, developers, and users.
This objective is crucial for the success of Independent Society of Knowledge, as it would provide the means for researchers to perform cutting-edge research, and for the community to build upon existing knowledge.

Another observation in different fields of computational research is that each library is design for the specific job of its developers, therefore, new algorithms, methods, and techniques are often not implemented in these libraries.
Our vision for our library ecosystem is to be able to provide the state-of-the-art algorithms, methods, and techniques in various fields of computational research, as soon as they get published, making our libraries efficient to use and up-to-date, as well as being complete.

\subsection{Education and Learning}\label{subsec:education-and-learning}
Education of the next generation of researchers, scientists, and contributors is crucial for the success of Independent Society of Knowledge.
We make sure to provide the essentials and foundations that would lead the community to open-education.
There are multiple projects in this field such as MIT Open-Courseware, Khan Academy, Coursera, edX, etc.
We seek collaboration and contribute to these ideas as well as promoting a more dynamical approach to education.

\subsection{Licenses and Attribution}\label{subsec:licenses-and-attribution}
One of the key objectives of Independent Society of Knowledge is to provide a set of licenses and attribution strategies that would allow for open-collaborative work, while also providing the means for contributors to be rewarded for their work.
These licenses and attribution strategies would be different from existing ones, such as Creative Commons, GNU General Public License, etc.
Our approach would be to create these licenses and attribution strategies in a way that they are compatible with academic and research endeavours.

Legal aspects of knowledge, maintenance, and attribution is an important mission for Independent Society of Knowledge, and we seek to not overlook this aspect.

\subsection{Finance and Sustainability}\label{subsec:finance-and-sustainability}
\begin{quotation}
    This part needs refinement
\end{quotation}
Financial sustainability is crucial for the success of Independent Society of Knowledge.
While we believe creating Open-Source software for science, research, and education is crucial, we preserve the fact that to make such infrastructure work and maintainable in long-term, we need to provide financial safety for us and our contributors.

Therefore, we look up to a financial model that is based on:
\begin{itemize}
    \item Donations and Grants: We seek donations from individuals, institutions, and organizations that support our vision and mission.
    We also look for grants from governmental and non-governmental organizations that support open-collaborative and decentralized initiatives.
    \item Memberships and Subscriptions: We provide memberships and subscriptions for individuals and institutions that want to support our vision and mission.
    These memberships and subscriptions would provide access to exclusive content, services, and features.
    \item Services and Consulting: We provide services and consulting for individuals, institutions, and organizations that want to implement our vision and mission in their own work.
    These services and consulting would provide expertise, guidance, and support.
\end{itemize}